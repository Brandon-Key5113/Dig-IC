\documentclass[11pt]{article}
\usepackage[margin=1.0in]{geometry}
\usepackage{mathtools}
\usepackage{graphicx}
\usepackage{float}
\usepackage[font=small]{caption}
\usepackage{color}
\usepackage{fancyhdr}
%\usepackage{soul} %for striking out text
%\usepackage{arydshln} % for dotted line on truth table
\usepackage{tikz} % graphics stuff
\usepackage{caption}
\usepackage{gensymb}
\usepackage{booktabs}
\usepackage{listings}
\usepackage{fancyvrb}
%\usepackage{subcaption} % used to horizontally tile figures
%\usepackage{listings}
\pagestyle{fancy}
\fancyhead{}
\fancyfoot{}
\fancyfoot[R]{\thepage}

\begin{document}

%
% cover page
%

\vspace*{2 cm}

\begin{center}
\bf{CMPE-630 Digital IC Design\\
    Laboratory Exercise 7\\
\vspace{0.25 cm}
Autolayout Design Techniques (HDL-Layout)
}
\end{center}

\vspace{6 cm}

\begin{flushright}
Brandon Key\\
Performed: 6 Nov 2019\\
Submitted: 13 Nov 2019\\
\vspace{0.5 cm}
Instructor: Dr. Amlan Ganguly\\
TAs: Abhishek Vashist\\
Andrew Fountain\\
Piers Kwan\\
\vspace{0.5 cm}
\end{flushright}

\vspace{3 cm}
\indent By submitting this report, you attest that you neither have given nor have received any assistance (including writing, collecting data, plotting figures, tables or graphs, or using previous student reports as a reference), and you further acknowledge that giving or receiving such assistance will result in a failing grade for this course.

\vspace{1 cm}
Your Signature:   \rule{13cm}{.1pt}


%\tableofcontents
\newpage

\section{Abstract}
	

\section{Design Methodology and Theory}

	\subsection{1-Bit ALU}	
	
		The 1 Bit ALU designed in this exercise was created from behavioral VHDL (see Listing \ref{lst:ALU\_1Bit). 
		Behavioral
		
		\begin{figure}[H]
			\centering
			\includegraphics[width=0.7\linewidth]{"Pictures/ALU-1Bit Schematic"}
			\caption{1 Bit ALU Schematic}
			\label{fig:alu-1bit-schematic}
		\end{figure}
		

	\subsection{n-Bit ALU}
	
		Structural
		
		
		\begin{figure}[H]
			\centering
			\includegraphics[width=0.7\linewidth]{"Pictures/ALU-16Bit Schematic 1"}
			\caption{16 Bit ALU Schematic Page 1}
			\label{fig:alu-16bit-schematic-1}
		\end{figure}
	
		\begin{figure}[H]
			\centering
			\includegraphics[width=0.7\linewidth]{"Pictures/ALU-16Bit Schematic 2"}
			\caption{16 Bit ALU Schematic Page 2}
			\label{fig:alu-16bit-schematic-2}
		\end{figure}
	
		\begin{figure}[H]
			\centering
			\includegraphics[width=0.7\linewidth]{"Pictures/ALU-16Bit Schematic 3"}
			\caption{16 Bit ALU Schematic Page 3}
			\label{fig:alu-16bit-schematic-3}
		\end{figure}


\section{Results and Analysis}

	\subsection{Functional Simulation}
	
		\subsubsection{1 Bit ALU}
		
			\begin{figure}[H]
				\centering
				\includegraphics[width=0.7\linewidth]{"Pictures/ALU_1Bit Functional Sim "}
				\caption{Functional Simulation of 1-bit ALU}
				\label{fig:alu1bit-functional-sim-}
			\end{figure}
		
			
	
		\subsubsection{16 Bit ALU}
		
			\begin{figure}[H]
				\centering
				\includegraphics[width=0.7\linewidth]{"Pictures/16 Bit ALU OR"}
				\caption{Functional Simulation of 16-bit ALU: OR}
				\label{fig:16-bit-alu-or}
			\end{figure}
			
			\begin{figure}[H]
				\centering
				\includegraphics[width=0.7\linewidth]{"Pictures/16 Bit ALU AND"}
				\caption{Functional Simulation of 16-bit ALU: AND}
				\label{fig:16-bit-alu-and}
			\end{figure}
		
			\begin{figure}[H]
				\centering
				\includegraphics[width=0.7\linewidth]{"Pictures/16 Bit ALU Add"}
				\caption{Functional Simulation of 16-bit ALU: Addition}
				\label{fig:16-bit-alu-add}
			\end{figure}

			\begin{figure}[H]
				\centering
				\includegraphics[width=0.7\linewidth]{"Pictures/16 Bit ALU Add Carry"}
				\caption{Functional Simulation of 16-bit ALU: Addition with carry}
				\label{fig:16-bit-alu-add-carry}
			\end{figure}
		
			
			\begin{figure}[H]
				\centering
				\includegraphics[width=0.7\linewidth]{"Pictures/16 Bit Alu Sub Pos"}
				\caption{Functional Simulation of 16-bit ALU: Subtraction}
				\label{fig:16-bit-alu-sub-pos}
			\end{figure}
			
	
			\begin{figure}[H]
				\centering
				\includegraphics[width=0.7\linewidth]{"Pictures/16 Bit ALU Sub Neg"}
				\caption{Functional Simulation of 16-bit ALU: Subtraction with negative result}
				\label{fig:16-bit-alu-sub-neg}
			\end{figure}
		
		
	\subsection{Layout}
	
		\subsubsection{1 Bit ALU}
		
			\begin{figure}[H]
				\centering
				\includegraphics[width=0.7\linewidth]{"Pictures/ALU-1Bit Layout"}
				\caption{1 Bit ALU Layout}
				\label{fig:alu-1bit-layout}
			\end{figure}
			
	
		\subsubsection{16 Bit ALU}
			
			Area 0.7
			
			Power Routing
			\begin{itemize}
				\item Varying levels of routing completion time
				\item Slight preference for jogs over via to fill the area.
				\item Rip
				\item Under rip options: 
				\subitem Rips Most Aggressive
				\subitem Automatic Rip Passes
				\subitem Reroute
				\item Under Advanced:
				\subitem Allow all directions for stubs
				\subitem Via Options > Use via generator
			\end{itemize}
		
			\begin{figure}[H]
				\centering
				\includegraphics[width=0.7\linewidth]{"Pictures/ALU-16Bit Layout"}
				\caption{16 Bit ALU Layout}
				\label{fig:alu-16bit-layout}
			\end{figure}
	
	\subsection{Timing}
	
		\subsubsection{1 Bit ALU}
			It was found that subtraction was by far the slowest operation, with the timing difference visible in the waveforms.  
		
			\begin{figure}[H]
				\centering
				\includegraphics[width=0.7\linewidth]{"Pictures/ALU_1Bit Timing"}
				\caption{1 Bit ALU Worst Case Timing Simulation}
				\label{fig:alu1bit-timing}
			\end{figure}
			
		
		\subsubsection{16 Bit ALU}
	
	\subsection{Power}
		
		\subsubsection{1 Bit ALU}
		
		\subsubsection{16 Bit ALU}

\section{Conclusion}


\section{Appendix}

	\subsection{VHDL}
	
		\lstinputlisting[language=VHDL, caption={Controller\_16Bit VHDL}, captionpos=tlabel{lst:Controller\_16Bit}]{"Source/Lab7_Alu/SourceCode/Controller_16Bit.vhd"}
		\lstinputlisting[language=VHDL, caption={nBitAdderSubtractor\_4Bit VHDL}, captionpos=tlabel{lst:nBitAdderSubtractor\_4Bit}]{"Source/Lab7_Alu/SourceCode/nBitAdderSubtractor_4Bit.vhd"}
		\lstinputlisting[language=VHDL, caption={FullAdder VHDL}, captionpos=tlabel{lst:FullAdder}]{"Source/Lab7_Alu/SourceCode/FullAdder.vhd"}
		\lstinputlisting[language=VHDL, caption={ALU\_16Bit\_tb VHDL}, captionpos=tlabel{lst:ALU\_16Bit\_tb}]{"Source/Lab7_Alu/SourceCode/ALU_16Bit_tb.vhd"}
		\lstinputlisting[language=VHDL, caption={Controller\_4Bit VHDL}, captionpos=tlabel{lst:Controller\_4Bit}]{"Source/Lab7_Alu/SourceCode/Controller_4Bit.vhd"}
		\lstinputlisting[language=VHDL, caption={nBitOR\_4Bit VHDL}, captionpos=tlabel{lst:nBitOR\_4Bit}]{"Source/Lab7_Alu/SourceCode/nBitOR_4Bit.vhd"}
		\lstinputlisting[language=VHDL, caption={ALU\_4Bit VHDL}, captionpos=tlabel{lst:ALU\_4Bit}]{"Source/Lab7_Alu/SourceCode/ALU_4Bit.vhd"}
		\lstinputlisting[language=VHDL, caption={nBitAND\_4Bit VHDL}, captionpos=tlabel{lst:nBitAND\_4Bit}]{"Source/Lab7_Alu/SourceCode/nBitAND_4Bit.vhd"}
		\lstinputlisting[language=VHDL, caption={ALU\_1Bit\_tb VHDL}, captionpos=tlabel{lst:ALU\_1Bit\_tb}]{"Source/Lab7_Alu/SourceCode/ALU_1Bit_tb.vhd"}
		\lstinputlisting[language=VHDL, caption={ALU\_1Bit VHDL}, captionpos=tlabel{lst:ALU\_1Bit}]{"Source/Lab7_Alu/SourceCode/ALU_1Bit.vhd"}
		\lstinputlisting[language=VHDL, caption={ALU\_4Bit\_tb VHDL}, captionpos=tlabel{lst:ALU\_4Bit\_tb}]{"Source/Lab7_Alu/SourceCode/ALU_4Bit_tb.vhd"}
		\lstinputlisting[language=VHDL, caption={ALU\_16Bit VHDL}, captionpos=tlabel{lst:ALU\_16Bit}]{"Source/Lab7_Alu/SourceCode/ALU_16Bit.vhd"}
		\lstinputlisting[language=VHDL, caption={nBitOR\_16Bit VHDL}, captionpos=tlabel{lst:nBitOR\_16Bit}]{"Source/Lab7_Alu/SourceCode/nBitOR_16Bit.vhd"}
		\lstinputlisting[language=VHDL, caption={nBitAdderSubtractor\_16Bit VHDL}, captionpos=tlabel{lst:nBitAdderSubtractor\_16Bit}]{"Source/Lab7_Alu/SourceCode/nBitAdderSubtractor_16Bit.vhd"}
		\lstinputlisting[language=VHDL, caption={nBitAND\_16Bit VHDL}, captionpos=tlabel{lst:nBitAND\_16Bit}]{"Source/Lab7_Alu/SourceCode/nBitAND_16Bit.vhd"}

	\subsection{SPICE}

		\lstinputlisting[caption={1-Bit ALU SPICE}, captionpos=tlabel{lst:ALU-1-Bit-Spice}]{"Source/Lab7_Alu/1 Bit ALU spice.cir"}

\end{document}
