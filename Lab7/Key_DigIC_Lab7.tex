\documentclass[11pt]{article}
\usepackage[margin=1.0in]{geometry}
\usepackage{mathtools}
\usepackage{graphicx}
\usepackage{float}
\usepackage[font=small]{caption}
\usepackage{color}
\usepackage{fancyhdr}
%\usepackage{soul} %for striking out text
%\usepackage{arydshln} % for dotted line on truth table
\usepackage{tikz} % graphics stuff
\usepackage{caption}
\usepackage{gensymb}
\usepackage{booktabs}
\usepackage{listings}
\usepackage{fancyvrb}
%\usepackage{subcaption} % used to horizontally tile figures
%\usepackage{listings}
\pagestyle{fancy}
\fancyhead{}
\fancyfoot{}
\fancyfoot[R]{\thepage}

\definecolor{mygreen}{rgb}{0,0.6,0}
\definecolor{mygray}{rgb}{0.5,0.5,0.5}
\definecolor{mymauve}{rgb}{0.58,0,0.82}

\lstset{ 
	backgroundcolor=\color{white},   % choose the background color; you must add \usepackage{color} or \usepackage{xcolor}; should come as last argument
	basicstyle=\footnotesize,        % the size of the fonts that are used for the code
	breakatwhitespace=false,         % sets if automatic breaks should only happen at whitespace
	breaklines=true,                 % sets automatic line breaking
	captionpos=t,                    % sets the caption-position to bottom
	commentstyle=\color{mygreen},    % comment style
	deletekeywords={},               % if you want to delete keywords from the given language
	escapeinside={\%*}{*)},          % if you want to add LaTeX within your code
	extendedchars=false,             % lets you use non-ASCII characters; for 8-bits encodings only, does not work with UTF-8
	firstnumber=0,                   % start line enumeration with line 1000
	frame=single,                    % adds a frame around the code
	keepspaces=true,                 % keeps spaces in text, useful for keeping indentation of code (possibly needs columns=flexible)
	keywordstyle=\color{blue},       % keyword style
	language=VHDL,                   % the language of the code
	morekeywords={},                 % if you want to add more keywords to the set
	numbers=left,                    % where to put the line-numbers; possible values are (none, left, right)
	numbersep=5pt,                   % how far the line-numbers are from the code
	numberstyle=\tiny\color{mygray}, % the style that is used for the line-numbers
	rulecolor=\color{black},         % if not set, the frame-color may be changed on line-breaks within not-black text (e.g. comments (green here))
	showspaces=false,                % show spaces everywhere adding particular underscores; it overrides 'showstringspaces'
	showstringspaces=false,          % underline spaces within strings only
	showtabs=false,                  % show tabs within strings adding particular underscores
	stepnumber=5,                    % the step between two line-numbers. If it's 1, each line will be numbered
	stringstyle=\color{mymauve},     % string literal style
	tabsize=4, 	                     % sets default tabsize to 2 spaces
	title=\lstname                   % show the filename of files included with \lstinputlisting; also try caption instead of title
}

\begin{document}

%
% cover page
%

\vspace*{2 cm}

\begin{center}
\bf{CMPE-630 Digital IC Design\\
    Laboratory Exercise 7\\
\vspace{0.25 cm}
Autolayout Design Techniques (HDL-Layout)
}
\end{center}

\vspace{6 cm}

\begin{flushright}
Brandon Key\\
Performed: 6 Nov 2019\\
Submitted: 13 Nov 2019\\
\vspace{0.5 cm}
Instructor: Dr. Amlan Ganguly\\
TAs: Abhishek Vashist\\
Andrew Fountain\\
Piers Kwan\\
\vspace{0.5 cm}
\end{flushright}

\vspace{3 cm}
\indent By submitting this report, you attest that you neither have given nor have received any assistance (including writing, collecting data, plotting figures, tables or graphs, or using previous student reports as a reference), and you further acknowledge that giving or receiving such assistance will result in a failing grade for this course.

\vspace{1 cm}
Your Signature:   \rule{13cm}{.1pt}


%\tableofcontents
\newpage

\section{Abstract}
	

\section{Design Methodology and Theory}

	The world of IC design is a very large and complex one. Engineering time is a critical factor when trying to create designs. An engineer tries to minimize power and cost, while maximizing performance. Modern ICs are very complex and manual engineer effort is not feasible. Fortunately, auto layout tools exists so that non-critical parts can be generated quickly. In this exercise, a single bit ALU and a 16-bit ALU were designed, automatically laid out and then power and timing results were extracted. 
	
	For all components designed in this exercise, VHDL was written to describe the functionality of the component. Leonardo Spectrum was used to turn the VHDL into synthesizable logic. The VHDL was then functionally tested with a test bench using Questa Sim. 
	
	

	\subsection{Functional Simulation}
	
		The ALUs in this exercise had a simple 2 bit op-code which can be seen in Table \ref{tab:OpCodes}.
	
		\begin{table}[H]
			\centering
			\caption{ALU Operations}
			\label{tab:OpCodes}
			\begin{tabular}{|ccc|}
				\hline
				\textbf{OpCode} & \textbf{Operation} & \textbf{Operands} \\
				\hline
				00              & AND                & A AND B           \\
				01              & OR                 & A OR B            \\
				10              & ADD                & A + B             \\
				11              & SUB                & A - B   \\   
				\hline       
			\end{tabular}
		\end{table}
	
		\subsubsection{1 Bit ALU}
		
			The 1 Bit ALU designed in this exercise was created from behavioral VHDL (see Listing \ref{lst:ALU-1Bit}). A Questa Sim simulation was performed to test the functionality of the 1 bit ALU. The test bench can be seen in Listing \ref{lst:ALU-1Bit-tb}. The test bench went through every op-code and every input. The resulting waveforms can be seen in Figure \ref{fig:alu1bit-functional-sim-}.
		
			\begin{figure}[H]
				\centering
				\includegraphics[width=1\linewidth]{"Pictures/ALU_1Bit Functional Sim "}
				\caption{Functional Simulation of 1-bit ALU}
				\label{fig:alu1bit-functional-sim-}
			\end{figure}
		
			The 1 Bit ALU functioned properly.
		
		\subsubsection{16 Bit ALU}
		
			The 16-bit ALU was created structurally with generically large structures. The VHDL that describes the ALU can be seen in Listing \ref{lst:ALU-16Bit}. While the 1-bit ALU could have a full test bench that tested every input, the 16-bit ALU was far too large to do the same. Instead, a few test cases were selected to test each function of the ALU. Each component in the ALU was previously tested, so the main goal was to test the setup of the ALU itself. The testbench code can be seen in Listing \ref{lst:ALU-16Bit-tb}. The AND functionality was tested first and captured in Figure \ref{fig:16-bit-alu-and}.
		
			\begin{figure}[H]
				\centering
				\includegraphics[width=0.7\linewidth]{"Pictures/16 Bit ALU AND"}
				\caption{Functional Simulation of 16-bit ALU: AND}
				\label{fig:16-bit-alu-and}
			\end{figure}
		
			Once the and functionality was verified, the OR functionality was tested and recorded in Figure \ref{fig:16-bit-alu-or}.
		
			\begin{figure}[H]
				\centering
				\includegraphics[width=0.7\linewidth]{"Pictures/16 Bit ALU OR"}
				\caption{Functional Simulation of 16-bit ALU: OR}
				\label{fig:16-bit-alu-or}
			\end{figure}
			
			x
			
			\begin{figure}[H]
				\centering
				\includegraphics[width=0.7\linewidth]{"Pictures/16 Bit ALU Add"}
				\caption{Functional Simulation of 16-bit ALU: Addition}
				\label{fig:16-bit-alu-add}
			\end{figure}
			
			\begin{figure}[H]
				\centering
				\includegraphics[width=0.7\linewidth]{"Pictures/16 Bit ALU Add Carry"}
				\caption{Functional Simulation of 16-bit ALU: Addition with carry}
				\label{fig:16-bit-alu-add-carry}
			\end{figure}
			
			
			\begin{figure}[H]
				\centering
				\includegraphics[width=0.7\linewidth]{"Pictures/16 Bit Alu Sub Pos"}
				\caption{Functional Simulation of 16-bit ALU: Subtraction}
				\label{fig:16-bit-alu-sub-pos}
			\end{figure}
			
			
			\begin{figure}[H]
				\centering
				\includegraphics[width=0.7\linewidth]{"Pictures/16 Bit ALU Sub Neg"}
				\caption{Functional Simulation of 16-bit ALU: Subtraction with negative result}
				\label{fig:16-bit-alu-sub-neg}
			\end{figure}

	\subsection{Schematic}

		\subsubsection{1-Bit ALU}	
		
			
			Behavioral
			
			\begin{figure}[H]
				\centering
				\includegraphics[width=0.4\linewidth]{"Pictures/ALU-1Bit Schematic"}
				\caption{1 Bit ALU Schematic}
				\label{fig:alu-1bit-schematic}
			\end{figure}
		

		\subsubsection{n-Bit ALU}
		
			Structural
			
			
			\begin{figure}[H]
				\centering
				\includegraphics[width=0.3\linewidth]{"Pictures/ALU-16Bit Schematic 1"}
				\caption{16 Bit ALU Schematic Page 1}
				\label{fig:alu-16bit-schematic-1}
			\end{figure}
		
			\begin{figure}[H]
				\centering
				\includegraphics[width=0.2\linewidth]{"Pictures/ALU-16Bit Schematic 2"}
				\caption{16 Bit ALU Schematic Page 2}
				\label{fig:alu-16bit-schematic-2}
			\end{figure}
		
			\begin{figure}[H]
				\centering
				\includegraphics[width=0.4\linewidth]{"Pictures/ALU-16Bit Schematic 3"}
				\caption{16 Bit ALU Schematic Page 3}
				\label{fig:alu-16bit-schematic-3}
			\end{figure}


\section{Results and Analysis}
		
		
	\subsection{Layout}
	
		\subsubsection{1 Bit ALU}
		
			\begin{figure}[H]
				\centering
				\includegraphics[width=0.7\linewidth]{"Pictures/ALU-1Bit Layout"}
				\caption{1 Bit ALU Layout}
				\label{fig:alu-1bit-layout}
			\end{figure}
			
	
		\subsubsection{16 Bit ALU}
			
			Area 0.7
			
			Power Routing
			\begin{itemize}
				\item Varying levels of routing completion time
				\item Slight preference for jogs over via to fill the area.
				\item Rip
				\item Under rip options: 
				\subitem Rips Most Aggressive
				\subitem Automatic Rip Passes
				\subitem Reroute
				\item Under Advanced:
				\subitem Allow all directions for stubs
				\subitem Via Options > Use via generator
			\end{itemize}
		
			\begin{figure}[H]
				\centering
				\includegraphics[width=0.7\linewidth]{"Pictures/ALU-16Bit Layout"}
				\caption{16 Bit ALU Layout}
				\label{fig:alu-16bit-layout}
			\end{figure}
	
	\subsection{Area}
	
		\subsubsection{1 Bit ALU}
	
		\subsubsection{16 Bit ALU}
	
	\subsection{Timing}
	
		\subsubsection{1 Bit ALU}
			It was found that subtraction was by far the slowest operation, with the timing difference visible in the waveforms.  
		
			\begin{figure}[H]
				\centering
				\includegraphics[width=1\linewidth]{"Pictures/ALU_1Bit Timing"}
				\caption{1 Bit ALU Worst Case Timing Simulation}
				\label{fig:alu1bit-timing}
			\end{figure}
			
			\begin{table}[H]
				\centering
				\caption{1-Bit ALU Worst Case Rise Time}
				\label{tab:ALU-1-Bit-Risetime}
				\begin{tabular}{|cclcr|}
					\hline
					\textbf{Output} & \textbf{Rise Time (ps)} & \textbf{A} & \textbf{B} & \textbf{Operation} \\
					Y               & 1167.5                  & 1          & 0          & SUB                \\
					Carry           & 1557.1                  & 0          & 1          & SUB                \\
					         \hline
				\end{tabular}
			\end{table}
		
		
			\begin{table}[H]
				\centering
				\caption{1-Bit ALU Worst Case Fall Time}
				\label{tab:ALU-1-Bit-Falltime}
				\begin{tabular}{|cclcr|}
					\hline
					\textbf{Output} & \textbf{Fall Time (ps)} & \textbf{A} & \textbf{B} & \textbf{Operation} \\
					Y               & 1001.3                  & 1          & 1          & SUB                \\
					Carry           & 1068.1                  & 1          & 1          & SUB                \\
					\hline
				\end{tabular}
			\end{table}
		
		
			\begin{table}[H]
				\centering
				\caption{1-Bit ALU Worst Case Propagation Time High to Low}
				\label{tab:ALU-1-Bit-Tpd-HL}
				\begin{tabular}{|cclcr|}
					\hline
					\textbf{Output} & \textbf{Tp,HL (ps)} & \textbf{A} & \textbf{B} & \textbf{Operation} \\
					Y               & 701.8                   & 1          & 1          & SUB                \\
					Carry           & 810.3                   & 1          & 1          & SUB                \\
					\hline
				\end{tabular}
			\end{table}
		
			\begin{table}[H]
				\centering
				\caption{1-Bit ALU Worst Case Propagation Time Low to High}
				\label{tab:ALU-1-Bit-Tpd-LH}
				\begin{tabular}{|cclcr|}
					\hline
					\textbf{Output} & \textbf{Tp,LH (ps)} & \textbf{A} & \textbf{B} & \textbf{Operation} \\
					Y               & 686.1                   & 1          & 0          & SUB                \\
					Carry           & 996.7                   & 0          & 1          & SUB                \\
					\hline
				\end{tabular}
			\end{table}
		
		\subsubsection{16 Bit ALU}
		
			Listing \ref{lst:ALU-16-Bit-Spice} shows the SPICE file used to simulate the 16-Bit ALU.  
		
		
			\begin{figure}[H]
				\centering
				\includegraphics[width=0.7\linewidth]{"Pictures/Terrible Simulation"}
				\caption{Simulation with Incorrect Forces}
				\label{fig:terrible-simulation}
			\end{figure}
			
		
			\begin{figure}[H]
				\centering
				\includegraphics[width=1\linewidth]{"Pictures/ALU 16-Bit Full Timing"}
				\caption{16-Bit ALU Timing Waveforms}
				\label{fig:alu-16-bit-full-timing}
			\end{figure}
			
		
			\begin{table}[H]
				\centering
				\caption{16-Bit ALU Worst Case Rise Time}
				\label{tab:ALU-16-Bit-Risetime}
				\begin{tabular}{|cclcrcr|}
					\hline
					\multicolumn{2}{|c}{\textbf{Input}} & \textbf{Output} & \multicolumn{4}{c|}{\textbf{Rise Time (ps)}} \\
					\textbf{A} & \textbf{B} & \textbf{Y} & \textbf{Op} & \textbf{Y} & \textbf{Op} & \textbf{CB} \\
					\hline
					0x0000 & 0x0000 & Y{[}15{]} & 11 & 938.0 & 11 & 1033.8 \\
					0xFFFF & 0xFFFF & Y{[}15{]} & 11 & 938.0 & 01 & 1124.5 \\
					0xFFFF & 0x0001 & Y{[}15{]} & 11 & 982.2 & 11 & 1044.6 \\
					0x0001 & 0xFFFF & Y{[}15{]} & 01 & 904.63 & 00 & 1044.6 \\
					0x0000 & 0x0001 & Y{[}0{]} & 00 & 1085.2 & 01 & 986.9 \\
					0xABCD & 0x89EF & Y{[}15{]} & 01 & 917.9 & 11 & 951.8 \\
					\hline
				\end{tabular}
			\end{table}
		
		
			\begin{table}[H]
				\centering
				\caption{16-Bit ALU Worst Case Fall Time}
				\label{tab:ALU-16-Bit-Falltime}
				\begin{tabular}{|cclcrcr|}
					\hline
					\multicolumn{2}{|c}{\textbf{Input}} & \textbf{Output} & \multicolumn{4}{c|}{\textbf{Fall Time (ps)}} \\
					\textbf{A} & \textbf{B} & \textbf{Y} & \textbf{Op} & \textbf{Y} & \textbf{Op} & \textbf{CB} \\
					\hline
					0x0000 & 0x0000 & Y{[}15{]} & 01 & 1100.3 & 01 & 1027.1 \\
					0xFFFF & 0xFFFF & Y{[}15{]} & 10 & 1067.0 & 10 & 1033.8 \\
					0xFFFF & 0x0001 & Y{[}15{]} & 11 & 830.6 & 01 & 1124.5 \\
					0x0001 & 0xFFFF & Y{[}15{]} & 10 & 1161.7 & 11 & 951.8 \\
					0x0000 & 0x0001 & Y{[}0{]} & 00 & 996.8 & 10 & 1005.9 \\
					0xABCD & 0x89EF & Y{[}15{]} & 01 & 855.5 & 01 & 969.5 \\
					\hline
				\end{tabular}
			\end{table}
		
		
			\begin{table}[H]
				\centering
				\caption{16-Bit ALU Worst Case Propagation Time High to Low}
				\label{tab:ALU-16-Bit-Tpd-HL}
				\begin{tabular}{|cclcrcr|}
					\hline
					\multicolumn{2}{|c}{\textbf{Input}} & \textbf{Output} & \multicolumn{4}{c|}{\textbf{Tp,HL (ps)}} \\
					\textbf{A} & \textbf{B} & \textbf{Y} & \textbf{Op} & \textbf{Y} & \textbf{Op} & \textbf{CB} \\
					\hline
					0x0000 & 0x0000 & Y{[}15{]} & 01 & 907.9 & 01 & 1154.5 \\
					0xFFFF & 0xFFFF & Y{[}15{]} & 10 & 2185.6 & 10 & 1281.1 \\
					0xFFFF & 0x0001 & Y{[}15{]} & 11 & 2154.6 & 01 & 1004.0 \\
					0x0001 & 0xFFFF & Y{[}15{]} & 10 & 1670.7 & 11 & 1143.7 \\
					0x0000 & 0x0001 & Y{[}0{]} & 00 & 1377.7 & 10 & 920.7 \\
					0xABCD & 0x89EF & Y{[}15{]} & 01 & 1465.4 & 01 & 872.0 \\
					\hline
				\end{tabular}
			\end{table}
		
		
			\begin{table}[H]
				\centering
				\caption{16-Bit ALU Worst Case Propagation Time Low to High}
				\label{tab:ALU-16-Bit-Tpd-LH}
				\begin{tabular}{|cclcrcr|}
					\hline
					\multicolumn{2}{|c}{\textbf{Input}} & \textbf{Output} & \multicolumn{4}{c|}{\textbf{Tp,LH (ps)}} \\
					\textbf{A} & \textbf{B} & \textbf{Y} & \textbf{Op} & \textbf{Y} & \textbf{Op} & \textbf{CB} \\
					\hline
					0x0000 & 0x0000 & Y{[}15{]} & 11 & 2185.6 & 11 & 955.2 \\
					0xFFFF & 0xFFFF & Y{[}15{]} & 11 & 2154.6 & 01 & 1521.9 \\
					0xFFFF & 0x0001 & Y{[}15{]} & 11 & 1670.7 & 11 & 1143.7 \\
					0x0001 & 0xFFFF & Y{[}15{]} & 01 & 1085.2 & 00 & 947.6 \\
					0x0000 & 0x0001 & Y{[}0{]} & 00 & 1099.2 & 01 & 1047.9 \\
					0xABCD & 0x89EF & Y{[}15{]} & 01 & 7608.9 & 11 & 1127.0 \\
					\hline
				\end{tabular}
			\end{table}
		
			
	
	\subsection{Power}
		
		\subsubsection{1 Bit ALU}
		
		\subsubsection{16 Bit ALU}

\section{Conclusion}


\section{Appendix}

	\subsection{VHDL}
	
		\lstinputlisting[caption={Controller-16Bit VHDL}\label{lst:Controller-16Bit}]{"Source/Lab7_Alu/SourceCode/Controller_16Bit.vhd"}
		\lstinputlisting[caption={nBitAdderSubtractor-4Bit VHDL}\label{lst:nBitAdderSubtractor-4Bit}]{"Source/Lab7_Alu/SourceCode/nBitAdderSubtractor_4Bit.vhd"}
		\lstinputlisting[caption={FullAdder VHDL}\label{lst:FullAdder}]{"Source/Lab7_Alu/SourceCode/FullAdder.vhd"}
		\lstinputlisting[caption={ALU-16Bit-tb VHDL}\label{lst:ALU-16Bit-tb}]{"Source/Lab7_Alu/SourceCode/ALU_16Bit_tb.vhd"}
		\lstinputlisting[caption={Controller-4Bit VHDL}\label{lst:Controller-4Bit}]{"Source/Lab7_Alu/SourceCode/Controller_4Bit.vhd"}
		\lstinputlisting[caption={nBitOR-4Bit VHDL}\label{lst:nBitOR-4Bit}]{"Source/Lab7_Alu/SourceCode/nBitOR_4Bit.vhd"}
		\lstinputlisting[caption={ALU-4Bit VHDL}\label{lst:ALU-4Bit}]{"Source/Lab7_Alu/SourceCode/ALU_4Bit.vhd"}
		\lstinputlisting[caption={nBitAND-4Bit VHDL}\label{lst:nBitAND-4Bit}]{"Source/Lab7_Alu/SourceCode/nBitAND_4Bit.vhd"}
		\lstinputlisting[caption={ALU-1Bit-tb VHDL}\label{lst:ALU-1Bit-tb}]{"Source/Lab7_Alu/SourceCode/ALU_1Bit_tb.vhd"}
		\lstinputlisting[caption={ALU-1Bit VHDL}\label{lst:ALU-1Bit}]{"Source/Lab7_Alu/SourceCode/ALU_1Bit.vhd"}
		\lstinputlisting[caption={ALU-4Bit-tb VHDL}\label{lst:ALU-4Bit-tb}]{"Source/Lab7_Alu/SourceCode/ALU_4Bit_tb.vhd"}
		\lstinputlisting[caption={ALU-16Bit VHDL}\label{lst:ALU-16Bit}]{"Source/Lab7_Alu/SourceCode/ALU_16Bit.vhd"}
		\lstinputlisting[caption={nBitOR-16Bit VHDL}\label{lst:nBitOR-16Bit}]{"Source/Lab7_Alu/SourceCode/nBitOR_16Bit.vhd"}
		\lstinputlisting[caption={nBitAdderSubtractor-16Bit VHDL}\label{lst:nBitAdderSubtractor-16Bit}]{"Source/Lab7_Alu/SourceCode/nBitAdderSubtractor_16Bit.vhd"}
		\lstinputlisting[caption={nBitAND-16Bit VHDL}\label{lst:nBitAND-16Bit}]{"Source/Lab7_Alu/SourceCode/nBitAND_16Bit.vhd"}

	\subsection{SPICE}

		\lstinputlisting[caption={1Bit ALU SPICE}\label{lst:ALU-1-Bit-Spice}]{"Source/Lab7_Alu/1 Bit ALU spice.cir"}
		
		\lstinputlisting[caption={6Bit ALU SPICE}\label{lst:ALU-16-Bit-Spice}]{"Source/Lab7_Alu/16 Bit ALU spice.cir"}

\end{document}
