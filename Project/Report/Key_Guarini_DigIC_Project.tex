\documentclass[11pt]{article}
\usepackage[margin=1.0in]{geometry}
\usepackage{lastpage}
\usepackage{mathtools}
\usepackage{graphicx}
\usepackage{float}
\usepackage[font=small]{caption}
\usepackage{color}
\usepackage{fancyhdr}
%\usepackage{soul} %for striking out text
%\usepackage{arydshln} % for dotted line on truth table
\usepackage{tikz} % graphics stuff
\usepackage{caption}
\usepackage{gensymb}
\usepackage{booktabs}
\usepackage{listings}
\usepackage{fancyvrb}
%\usepackage{subcaption} % used to horizontally tile figures
%\usepackage{listings}
\pagestyle{fancy}
\fancyhead{}
\fancyfoot{}


\newcommand{\ClassNameS}{CMPE-630}
\newcommand{\ClassNameL}{Digital Integrated Circuit Design}
\newcommand{\ExerciseName}{Final Project}
\newcommand{\SubTitle}{Multiply and Accumulate (MAC) Datapath Unit Design}
\newcommand{\Arthur}{Brandon Key \& Chris Guarini}
\newcommand{\DueDate}{9 Dec 2019}

\fancyhead[L]{\ClassNameS}
\fancyhead[R]{\ExerciseName}

\fancyfoot[L]{\Arthur}
\fancyfoot[C]{\DueDate}
\fancyfoot[R]{\thepage \space of \pageref{LastPage}}

\setlength{\parindent}{0em}
\setlength{\parskip}{1em}

\linespread{1.0}
% where value determine line spacing.
% 1.0	single spacing
% 1.3	one-and-a-half spacing
% 1.6	double spacing

\definecolor{mygreen}{rgb}{0,0.6,0}
\definecolor{mygray}{rgb}{0.5,0.5,0.5}
\definecolor{mymauve}{rgb}{0.58,0,0.82}

\lstset{ 
	backgroundcolor=\color{white},   % choose the background color; you must add \usepackage{color} or \usepackage{xcolor}; should come as last argument
	basicstyle=\footnotesize,        % the size of the fonts that are used for the code
	breakatwhitespace=false,         % sets if automatic breaks should only happen at whitespace
	breaklines=true,                 % sets automatic line breaking
	captionpos=t,                    % sets the caption-position to bottom
	commentstyle=\color{mygreen},    % comment style
	deletekeywords={},               % if you want to delete keywords from the given language
	escapeinside={\%*}{*)},          % if you want to add LaTeX within your code
	extendedchars=false,             % lets you use non-ASCII characters; for 8-bits encodings only, does not work with UTF-8
	firstnumber=0,                   % start line enumeration with line 1000
	frame=single,                    % adds a frame around the code
	keepspaces=true,                 % keeps spaces in text, useful for keeping indentation of code (possibly needs columns=flexible)
	keywordstyle=\color{blue},       % keyword style
	language=VHDL,                   % the language of the code
	morekeywords={},                 % if you want to add more keywords to the set
	numbers=left,                    % where to put the line-numbers; possible values are (none, left, right)
	numbersep=5pt,                   % how far the line-numbers are from the code
	numberstyle=\tiny\color{mygray}, % the style that is used for the line-numbers
	rulecolor=\color{black},         % if not set, the frame-color may be changed on line-breaks within not-black text (e.g. comments (green here))
	showspaces=false,                % show spaces everywhere adding particular underscores; it overrides 'showstringspaces'
	showstringspaces=false,          % underline spaces within strings only
	showtabs=false,                  % show tabs within strings adding particular underscores
	stepnumber=5,                    % the step between two line-numbers. If it's 1, each line will be numbered
	stringstyle=\color{mymauve},     % string literal style
	tabsize=4, 	                     % sets default tabsize to 2 spaces
	title=\lstname                   % show the filename of files included with \lstinputlisting; also try caption instead of title
}

\begin{document}\thispagestyle{empty}

%
% cover page
%

\vspace*{2 cm}

\begin{center}
\bf{\ClassNameS \space \ClassNameL\\
    \ExerciseName\\
\vspace{0.25 cm}
\SubTitle
}
\end{center}

\vspace{4 cm}

\begin{flushright}
Brandon Key\\
Chris Guarini\\
Performed: 9 Dec 2019\\
Submitted: \DueDate\\
\vspace{0.5 cm}
Instructor: Dr. Amlan Ganguly\\
TAs: Abhishek Vashist\\
Andrew Fountain\\
Piers Kwan\\
\vspace{0.5 cm}
\end{flushright}

\vspace{3 cm}
\indent By submitting this report, you attest that you neither have given nor have received any assistance (including writing, collecting data, plotting figures, tables or graphs, or using previous student reports as a reference), and you further acknowledge that giving or receiving such assistance will result in a failing grade for this course.

\vspace{1 cm}
Your Signature:   \rule{13cm}{.1pt}

\newpage
\tableofcontents
\newpage

\section{Abstract}

	
	

\section{Design Methodology and Theory}

	A cornerstone of IC design is the ability to create large, complex designs from smaller more manageable parts. The project outlined in this exercise calls for the design, testing and layout of a multiply and accumulate (MAC) unit, which takes two 16-bit inputs, multiplies them together, adds them to the value stored in a register, and then stores that output back into the register. The final component should contain a built in self test (BIST) that verifies the functionality of the MAC.
	
	The MAC is composed of a carry-save multiplier, ripple carry full-adder, and parallel register. The BIST is implemented through the use of an LFSR for the inputs, an MISR for the output, and a test controller which controls the timing and sets the test passed and test complete outputs. A full diagram of the MAC with BIST can be seen below in Figure \ref{fig:full-project-block}.
	
	\begin{figure}[H]
		\centering
		\includegraphics[width=0.7\linewidth]{Pictures/Full-Project-Block}
		\caption{High Level Block Diagram of the MAC with BIST.}
		\label{fig:full-project-block}
	\end{figure}


	\subsection{User Operation}
		The full MAC and BIST design has a total of six inputs and three outputs, which are shown in Table \ref{tab:Inputs} and Table \ref{tab:Outputs} respectively. All inputs and outputs of the MAC are clocked, so a clk signal is required for operation. To enable the writing to register, WE must be high for both MAC and BIST functionality. This allows for initialization of input values without it writing to the registers, and for the ability to hold what value the registers contain without changing the inputs. 
		
		In order to run the MAC in its functional mode, WE must be high, reset and StartTest must be low. Once these requirements are met, the values of inputs A and B will be multiplied together and accumulated into the registers. The output RegOut will contain the current value in the register.
		
		To run the BIST, the component must first be reset for a total of three clock cycles during initializing. This can be achieved by holding reset low during initialization. Afterwords, reset, WE and StartTest should be high. The BIST runs for a total of 1000 clock cycles, during which the outputs Pass and Complete will be low. When Complete goes high the BIST is over and Pass will contain whether the test passed or not.
		
		\begin{table}[H]
			\centering
			\caption{Inputs of the MAC}
			\label{tab:Inputs}
			\begin{tabular}{|ccc|}
				\hline
				\textbf{}   \textbf{Input}      & \textbf{Function} &  \textbf{Size (Bits)} \\
				\hline
				\textbf{A}  & Input 1 & N/2          \\
				\textbf{B}  & Input 2 & N/2            \\ 
				\textbf{clk}  & Clock Signal & 1        \\ 
				\textbf{WE}  & Write Enable & 1           \\ 
				\textbf{reset}  & Active Low Reset & 1           \\ 
				\textbf{StartTest}  & Enable BIST & 1           \\ 
				\hline                     
			\end{tabular}
		\end{table}
		
		\begin{table}[H]
			\centering
			\caption{Outputs of the MAC}
			\label{tab:Outputs}
			\begin{tabular}{|ccc|}
				\hline
				\textbf{}   \textbf{Output}      & \textbf{Function} &  \textbf{Size (Bits)} \\
				\hline
				\textbf{RegOut}  & Output of the MAC & N          \\
				\textbf{Pass}  & BIST Pass Flag & 1            \\ 
				\textbf{Complete}  & BIST Complete Flag & 1        \\ 
				\hline                     
			\end{tabular}
		\end{table}
	

	\subsection{Adder}
	
		The adder used in this design is an N-bit ripple carry adder which accepts input from the outputs of the multiplier and accumulation register. A ripple carry adder is composed of a series of N full adders, where the carry-out of the previous full adder is fed into the carry-in of the next full adder, as shown in Figure \ref{fig:ripple-carry-adder}.
		
		\begin{figure}[H]
			\centering
			\includegraphics[width=0.6\textwidth,height=\dimexpr\textheight-4\baselineskip-\abovecaptionskip-\belowcaptionskip\relax,keepaspectratio]{"Pictures/Ripple Carry Adder"}
			\caption{Ripple Carry Adder}
			\label{fig:ripple-carry-adder}
		\end{figure}
		
		The code in Figure \ref{code:nBitAdder} shows one of the generate statements used to create the nBitAdder from full adder components. The uses VHDL generate statements to generate N-2 full adders that take their carry-in bit from the previous full adder and send their carry out to the next full adder through the internal signal carray. The first and last full adder in the series are generated differently, as the first full adder's carry-in is hard coded to 0 and the last full adder's carry-out is an output. The full code of the nBitAdder is available in Listing \ref{lst:nBitAdder-vhd} of the appendix.
		
		\begin{figure}[H]
		\centering
		\begin{verbatim}
            i_mid : if (i /= 0) and (i /= (n-1)) generate
                adder : full_adder port map(
                    A => A(i),
                    B => B(i),
                    Cin => c_array(i-1),
                    Sum => Y(i),
                    Cout => c_array(i)
                );
            end generate i_mid;
        \end{verbatim}
        \caption{nBitAdder Code Snippet} 
    	\label{code:nBitAdder} 
    	\end{figure}
	
		\begin{figure}[H] 
			\centering 
			\includegraphics[width=\textwidth,height=\dimexpr\textheight-4\baselineskip-\abovecaptionskip-\belowcaptionskip\relax,keepaspectratio]{"Pictures/nBitAdder Schematic Page 1"}
			\caption{nBitAdder Schematic Page 1} 
			\label{fig:nBitAdder-Schematic-Page-1} 
		\end{figure}
		
		
		\begin{figure}[H] 
			\centering 
			\includegraphics[width=\textwidth,height=\dimexpr\textheight-4\baselineskip-\abovecaptionskip-\belowcaptionskip\relax,keepaspectratio]{"Pictures/nBitAdder Schematic Page 2"}
			\caption{nBitAdder Schematic Page 2} 
			\label{fig:nBitAdder-Schematic-Page-2} 
		\end{figure}
	
	\subsection{Multiplier}
	
	The MAC includes a carry-save multiplier component in order to multiply the two inputs together. A carry-save multiplier works as shown in Figure \ref{fig:multiplier-block-dia}, where full adders are arranged in a 2d array where each row is N/2 adders long and offset by 1.
		
		\begin{figure}[H]
			\centering
			\includegraphics[width=\textwidth,height=\dimexpr\textheight-4\baselineskip-\abovecaptionskip-\belowcaptionskip\relax,keepaspectratio]{Pictures/Multiplier}
			\caption{Carry-Save Multiplier Block Diagram}
			\label{fig:multiplier-block-dia}
		\end{figure}
		
	The full VHDL code for the multiplier is found in Listing \ref{lst:Multiplier-vhd}. The code takes advantage of VHDL generate statements to generate the four different connections a full adder could have in a carry-save multiplier. The first fill adder of each row, the last full adder of the first row, the last full adder of every other row, and all other full adders.
	
	Figures \ref{fig:Multiplier-Schematic-Page-1} and \ref{fig:Multiplier-Schematic-Page-2} show both pages of the multiplier schematic. These schematics were generated from the VHDL code in \ref{lst:Multiplier-vhd} using Spectrum scripts to create a Verilog file that was then imported into Pyxis.
	
	
		\begin{figure}[H] 
			\centering 
			\includegraphics[width=\textwidth,height=\dimexpr\textheight-4\baselineskip-\abovecaptionskip-\belowcaptionskip\relax,keepaspectratio]{"Pictures/Multiplier Schematic Page 1"}
			\caption{Multiplier Schematic Page 1} 
			\label{fig:Multiplier-Schematic-Page-1} 
		\end{figure}
		
		
		\begin{figure}[H] 
			\centering 
			\includegraphics[width=\textwidth,height=\dimexpr\textheight-4\baselineskip-\abovecaptionskip-\belowcaptionskip\relax,keepaspectratio]{"Pictures/Multiplier Schematic Page 2"}
			\caption{Multiplier Schematic Page 2} 
			\label{fig:Multiplier-Schematic-Page-2} 
		\end{figure}
	
	\subsection{16-Bit Register}
	
	Two 16-bit parallel registers are used to clock the two inputs to the multiplier. The VHDL architecture for the register can be seen in Figure \ref{code:nBitRegister_16}, which shows that the register has an active low reset, and when WE is high will take new input. Otherwise the register holds its previous value.
	
	\begin{figure}[H]
		\centering
		\begin{verbatim}
            architecture behav of nBitRegister_16 is
            begin
                output_proc : process (clk, Reset) begin
                    if Reset = '0' then
                        Y <= (others => '0');
                    elsif clk'event and clk = '1' then
                        if WE = '1' then
                            Y <= nBitIn;
                        end if;
                    end if;
                end process output_proc;
            end behav;
        \end{verbatim}
        \caption{16-Bit Register Code Snippet} 
    	\label{code:nBitRegister_16} 
    \end{figure}
    
    Figure \ref{fig:nBitRegister-16-Bit-Schematic} shows the schematic for the 16-bit register, which provides some insight into how a parallel register is made. There are 16 flip-flops which are connected to their respective input bits, the inverse of the reset signal, a write enable signal, and the clock. Their outputs are connected to the respective output bit.

		\begin{figure}[H] 
			\centering 
			\includegraphics[width=\textwidth,height=\dimexpr\textheight-4\baselineskip-\abovecaptionskip-\belowcaptionskip\relax,keepaspectratio]{"Pictures/nBitRegister 16-Bit Schematic"}
			\caption{16 Bit Register Schematic} 
			\label{fig:nBitRegister-16-Bit-Schematic} 
		\end{figure}

	\subsection{32-Bit Register}
		
		A 32-bit parallel register is used as the accumulator in the MAC component. The accumulator register if functionally the same as the 16-bit register above, as they are generated from the same generic VHDL code, just with different values for N.
		
		The code in Figure \ref{code:nBitRegister_32} shows the code used to generate the 32-bit register, which is identical to the 16-bit register in all but name. The schematic below in Figure \ref{fig:nBitRegister-32-Bit-Schematic} shows how the 32-bit register is functionally identical to the 16-bit register, just with double the amount of flip-flops and logic.
		
		\begin{figure}[H]
		\centering
		\begin{verbatim}
        architecture behav of nBitRegister_32 is
        begin
            output_proc : process (clk, Reset) begin
                if Reset = '0' then
                    Y <= (others => '0');
                elsif clk'event and clk = '1' then
                    if WE = '1' then
                        Y <= nBitIn;
                    end if;
                end if;
            end process output_proc;
        end behav;
        \end{verbatim}
        \caption{32-Bit Register Code Snippet} 
    	\label{code:nBitRegister_32} 
    \end{figure}
	
		\begin{figure}[H] 
			\centering 
			\includegraphics[width=\textwidth,height=\dimexpr\textheight-4\baselineskip-\abovecaptionskip-\belowcaptionskip\relax,keepaspectratio]{"Pictures/nBitRegister 32-Bit Schematic"}
			\caption{32 Bit Register Schematic} 
			\label{fig:nBitRegister-32-Bit-Schematic} 
		\end{figure}
	
	\subsection{MAC}
	
		\begin{figure}[H] 
			\centering 
			\includegraphics[width=\textwidth,height=\dimexpr\textheight-4\baselineskip-\abovecaptionskip-\belowcaptionskip\relax,keepaspectratio]{"Pictures/MAC-block"}
			\caption{MAC block} 
			\label{fig:MAC-block} 
		\end{figure}
		
		\begin{figure}[H] 
			\centering 
			\includegraphics[width=\textwidth,height=\dimexpr\textheight-4\baselineskip-\abovecaptionskip-\belowcaptionskip\relax,keepaspectratio]{"Pictures/MAC-16bit-Test-Bench"}
			\caption{MAC 16bit Test Bench} 
			\label{fig:MAC-16bit-Test-Bench} 
		\end{figure}
		
		
		\begin{figure}[H] 
			\centering 
			\includegraphics[width=\textwidth,height=\dimexpr\textheight-4\baselineskip-\abovecaptionskip-\belowcaptionskip\relax,keepaspectratio]{"Pictures/MAC-Test-Bench"}
			\caption{MAC Test Bench} 
			\label{fig:MAC-Test-Bench} 
		\end{figure}
	
	
	\subsection{Mux}
		The multiplexer was used to change the input from the user input to the 
	
		\begin{figure}[H] 
			\centering 
			\includegraphics[width=\textwidth,height=\dimexpr\textheight-4\baselineskip-\abovecaptionskip-\belowcaptionskip\relax,keepaspectratio]{"Pictures/nBitMux_2to1 Schematic"}
			\caption{nBitMux 2to1 Schematic} 
			\label{fig:nBitMux-2to1-Schematic} 
		\end{figure}
	
	\subsection{LFSR}
	
		Pseudo random number generator with known pattern.
		
		taps at 32,20,26,25
		//TODO
		
		\begin{figure}[H]
			\centering
			\includegraphics[width=\textwidth,height=\dimexpr\textheight-4\baselineskip-\abovecaptionskip-\belowcaptionskip\relax,keepaspectratio]{Pictures/LFSR}
			\caption{8-Bit Linear Feedback Shift Register}
			\label{fig:lfsr-block}
		\end{figure}
	
	\subsection{MISR}
	
	
		\begin{figure}[H]
			\centering
			\includegraphics[width=\textwidth,height=\dimexpr\textheight-4\baselineskip-\abovecaptionskip-\belowcaptionskip\relax,keepaspectratio]{Pictures/MISR}
			\caption{Multiple Input Shift Register}
			\label{fig:misr-block}
		\end{figure}
	
	\subsection{BIST}
	
		//TODO Talk about funcitonality
	
		\begin{figure}[H] 
			\centering 
			\includegraphics[width=\textwidth,height=\dimexpr\textheight-4\baselineskip-\abovecaptionskip-\belowcaptionskip\relax,keepaspectratio]{"Pictures/BIST-Test-Bench"}
			\caption{BIST Test Bench} 
			\label{fig:BIST-Test-Bench} 
		\end{figure}
	
	\subsection{Schematic}
	
	
		
		
		\begin{figure}[H] 
			\centering 
			\includegraphics[width=\textwidth,height=\dimexpr\textheight-4\baselineskip-\abovecaptionskip-\belowcaptionskip\relax,keepaspectratio]{"Pictures/Full Schematic Page 1"}
			\caption{Full Schematic Page 1} 
			\label{fig:Full-Schematic-Page-1} 
		\end{figure}
	
		\begin{figure}[H] 
			\centering 
			\includegraphics[width=\textwidth,height=\dimexpr\textheight-4\baselineskip-\abovecaptionskip-\belowcaptionskip\relax,keepaspectratio]{"Pictures/Full Schematic Page 2"}
			\caption{Full Schematic Page 2} 
			\label{fig:Full-Schematic-Page-2} 
		\end{figure}
	
		\begin{figure}[H] 
			\centering 
			\includegraphics[width=\textwidth,height=\dimexpr\textheight-4\baselineskip-\abovecaptionskip-\belowcaptionskip\relax,keepaspectratio]{"Pictures/Full Schematic Page 3"}
			\caption{Full Schematic Page 3} 
			\label{fig:Full-Schematic-Page-3} 
		\end{figure}
		
		
		\begin{figure}[H] 
			\centering 
			\includegraphics[width=\textwidth,height=\dimexpr\textheight-4\baselineskip-\abovecaptionskip-\belowcaptionskip\relax,keepaspectratio]{"Pictures/Full Schematic Page 4"}
			\caption{Full Schematic Page 4} 
			\label{fig:Full-Schematic-Page-4} 
		\end{figure}
		
		
		\begin{figure}[H] 
			\centering 
			\includegraphics[width=\textwidth,height=\dimexpr\textheight-4\baselineskip-\abovecaptionskip-\belowcaptionskip\relax,keepaspectratio]{"Pictures/Full Schematic Page 5"}
			\caption{Full Schematic Page 5} 
			\label{fig:Full-Schematic-Page-5} 
		\end{figure}
		
		
		\begin{figure}[H] 
			\centering 
			\includegraphics[width=\textwidth,height=\dimexpr\textheight-4\baselineskip-\abovecaptionskip-\belowcaptionskip\relax,keepaspectratio]{"Pictures/Full Schematic Page 6"}
			\caption{Full Schematic Page 6} 
			\label{fig:Full-Schematic-Page-6} 
		\end{figure}
		
		
		\begin{figure}[H] 
			\centering 
			\includegraphics[width=\textwidth,height=\dimexpr\textheight-4\baselineskip-\abovecaptionskip-\belowcaptionskip\relax,keepaspectratio]{"Pictures/Full Schematic Page 7"}
			\caption{Full Schematic Page 7} 
			\label{fig:Full-Schematic-Page-7} 
		\end{figure}
		
		
		\begin{figure}[H] 
			\centering 
			\includegraphics[width=\textwidth,height=\dimexpr\textheight-4\baselineskip-\abovecaptionskip-\belowcaptionskip\relax,keepaspectratio]{"Pictures/Full Schematic Page 8"}
			\caption{Full Schematic Page 8} 
			\label{fig:Full-Schematic-Page-8} 
		\end{figure}
		
		
		\begin{figure}[H] 
			\centering 
			\includegraphics[width=\textwidth,height=\dimexpr\textheight-4\baselineskip-\abovecaptionskip-\belowcaptionskip\relax,keepaspectratio]{"Pictures/Full Schematic Page 9"}
			\caption{Full Schematic Page 9} 
			\label{fig:Full-Schematic-Page-9} 
		\end{figure}
		
		
		\begin{figure}[H] 
			\centering 
			\includegraphics[width=\textwidth,height=\dimexpr\textheight-4\baselineskip-\abovecaptionskip-\belowcaptionskip\relax,keepaspectratio]{"Pictures/Full Schematic Page 10"}
			\caption{Full Schematic Page 10} 
			\label{fig:Full-Schematic-Page-10} 
		\end{figure}
		
		
		\begin{figure}[H] 
			\centering 
			\includegraphics[width=\textwidth,height=\dimexpr\textheight-4\baselineskip-\abovecaptionskip-\belowcaptionskip\relax,keepaspectratio]{"Pictures/Full Schematic Page 11"}
			\caption{Full Schematic Page 11} 
			\label{fig:Full-Schematic-Page-11} 
		\end{figure}
		
		
		\begin{figure}[H] 
			\centering 
			\includegraphics[width=\textwidth,height=\dimexpr\textheight-4\baselineskip-\abovecaptionskip-\belowcaptionskip\relax,keepaspectratio]{"Pictures/Full Schematic Page 12"}
			\caption{Full Schematic Page 12} 
			\label{fig:Full-Schematic-Page-12} 
		\end{figure}
		
		
		\begin{figure}[H] 
			\centering 
			\includegraphics[width=\textwidth,height=\dimexpr\textheight-4\baselineskip-\abovecaptionskip-\belowcaptionskip\relax,keepaspectratio]{"Pictures/Full Schematic Page 13"}
			\caption{Full Schematic Page 13} 
			\label{fig:Full-Schematic-Page-13} 
		\end{figure}
		
		
		\begin{figure}[H] 
			\centering 
			\includegraphics[width=\textwidth,height=\dimexpr\textheight-4\baselineskip-\abovecaptionskip-\belowcaptionskip\relax,keepaspectratio]{"Pictures/Full Schematic Page 14"}
			\caption{Full Schematic Page 14} 
			\label{fig:Full-Schematic-Page-14} 
		\end{figure}
		
		
		\begin{figure}[H] 
			\centering 
			\includegraphics[width=\textwidth,height=\dimexpr\textheight-4\baselineskip-\abovecaptionskip-\belowcaptionskip\relax,keepaspectratio]{"Pictures/Full Schematic Page 15"}
			\caption{Full Schematic Page 15} 
			\label{fig:Full-Schematic-Page-15} 
		\end{figure}
		
		
		\begin{figure}[H] 
			\centering 
			\includegraphics[width=\textwidth,height=\dimexpr\textheight-4\baselineskip-\abovecaptionskip-\belowcaptionskip\relax,keepaspectratio]{"Pictures/Full Schematic Page 16"}
			\caption{Full Schematic Page 16} 
			\label{fig:Full-Schematic-Page-16} 
		\end{figure}
		
		
		\begin{figure}[H] 
			\centering 
			\includegraphics[width=\textwidth,height=\dimexpr\textheight-4\baselineskip-\abovecaptionskip-\belowcaptionskip\relax,keepaspectratio]{"Pictures/Full Schematic Page 17"}
			\caption{Full Schematic Page 17} 
			\label{fig:Full-Schematic-Page-17} 
		\end{figure}
		
		
		\begin{figure}[H] 
			\centering 
			\includegraphics[width=\textwidth,height=\dimexpr\textheight-4\baselineskip-\abovecaptionskip-\belowcaptionskip\relax,keepaspectratio]{"Pictures/Full Schematic Page 18"}
			\caption{Full Schematic Page 18} 
			\label{fig:Full-Schematic-Page-18} 
		\end{figure}
		
		
		\begin{figure}[H] 
			\centering 
			\includegraphics[width=\textwidth,height=\dimexpr\textheight-4\baselineskip-\abovecaptionskip-\belowcaptionskip\relax,keepaspectratio]{"Pictures/Full Schematic Page 19"}
			\caption{Full Schematic Page 19} 
			\label{fig:Full-Schematic-Page-19} 
		\end{figure}
		
		
		\begin{figure}[H] 
			\centering 
			\includegraphics[width=\textwidth,height=\dimexpr\textheight-4\baselineskip-\abovecaptionskip-\belowcaptionskip\relax,keepaspectratio]{"Pictures/Full Schematic Page 20"}
			\caption{Full Schematic Page 20} 
			\label{fig:Full-Schematic-Page-20} 
		\end{figure}
		
		
		\begin{figure}[H] 
			\centering 
			\includegraphics[width=\textwidth,height=\dimexpr\textheight-4\baselineskip-\abovecaptionskip-\belowcaptionskip\relax,keepaspectratio]{"Pictures/Full Schematic Page 21"}
			\caption{Full Schematic Page 21} 
			\label{fig:Full-Schematic-Page-21} 
		\end{figure}
		
		
		\begin{figure}[H] 
			\centering 
			\includegraphics[width=\textwidth,height=\dimexpr\textheight-4\baselineskip-\abovecaptionskip-\belowcaptionskip\relax,keepaspectratio]{"Pictures/Full Schematic Page 22"}
			\caption{Full Schematic Page 22} 
			\label{fig:Full-Schematic-Page-22} 
		\end{figure}
		
		
		\begin{figure}[H] 
			\centering 
			\includegraphics[width=\textwidth,height=\dimexpr\textheight-4\baselineskip-\abovecaptionskip-\belowcaptionskip\relax,keepaspectratio]{"Pictures/Full Schematic Page 23"}
			\caption{Full Schematic Page 23} 
			\label{fig:Full-Schematic-Page-23} 
		\end{figure}
		
		
		\begin{figure}[H] 
			\centering 
			\includegraphics[width=\textwidth,height=\dimexpr\textheight-4\baselineskip-\abovecaptionskip-\belowcaptionskip\relax,keepaspectratio]{"Pictures/Full Schematic Page 24"}
			\caption{Full Schematic Page 24} 
			\label{fig:Full-Schematic-Page-24} 
		\end{figure}
		
		
		\begin{figure}[H] 
			\centering 
			\includegraphics[width=\textwidth,height=\dimexpr\textheight-4\baselineskip-\abovecaptionskip-\belowcaptionskip\relax,keepaspectratio]{"Pictures/Full Schematic Page 25"}
			\caption{Full Schematic Page 25} 
			\label{fig:Full-Schematic-Page-25} 
		\end{figure}
		
		
		\begin{figure}[H] 
			\centering 
			\includegraphics[width=\textwidth,height=\dimexpr\textheight-4\baselineskip-\abovecaptionskip-\belowcaptionskip\relax,keepaspectratio]{"Pictures/Full Schematic Page 26"}
			\caption{Full Schematic Page 26} 
			\label{fig:Full-Schematic-Page-26} 
		\end{figure}
		
		
		\begin{figure}[H] 
			\centering 
			\includegraphics[width=\textwidth,height=\dimexpr\textheight-4\baselineskip-\abovecaptionskip-\belowcaptionskip\relax,keepaspectratio]{"Pictures/Full Schematic Page 27"}
			\caption{Full Schematic Page 27} 
			\label{fig:Full-Schematic-Page-27} 
		\end{figure}
		
		
		\begin{figure}[H] 
			\centering 
			\includegraphics[width=\textwidth,height=\dimexpr\textheight-4\baselineskip-\abovecaptionskip-\belowcaptionskip\relax,keepaspectratio]{"Pictures/Full Schematic Page 28"}
			\caption{Full Schematic Page 28} 
			\label{fig:Full-Schematic-Page-28} 
		\end{figure}
		
		

		
		
		

\section{Results and Analysis}
		
	The MAC was initially laid out structurally; components were laid out and turned into cells that would then be connected together. This was done to limit the complexity of the final design. 
	
	
	\subsection{Components}
	
		The full adder was laid out first. The resulting layout can be seen in Figure \ref{fig:Full-Adder-Layout}.
		
		\begin{figure}[H] 
			\centering 
			\includegraphics[width=\textwidth,height=\dimexpr\textheight-4\baselineskip-\abovecaptionskip-\belowcaptionskip\relax,keepaspectratio]{"Pictures/Full Adder Layout"}
			\caption{Full Adder Layout} 
			\label{fig:Full-Adder-Layout} 
		\end{figure}
	
		The multiplier was a complex component so the feasibility of layout was questioned early. It turns out that giving appropriate area to run wires make routing the multiplier easy. The results can be seen in Figure \ref{fig:Multiplier-Layout}.
	
		\begin{figure}[H] 
			\centering 
			\includegraphics[width=\textwidth,height=\dimexpr\textheight-4\baselineskip-\abovecaptionskip-\belowcaptionskip\relax,keepaspectratio]{"Pictures/Multiplier Layout"}
			\caption{Multiplier Layout} 
			\label{fig:Multiplier-Layout} 
		\end{figure}
	
		The inputs of the multiplier needed to be controlled so the 16-bit register was laid out and captured in Figure \ref{fig:nBitRegister-16-Bit-Layout}.
		
		\begin{figure}[H] 
			\centering 
			\includegraphics[width=\textwidth,height=\dimexpr\textheight-4\baselineskip-\abovecaptionskip-\belowcaptionskip\relax,keepaspectratio]{"Pictures/nBitRegister 16-Bit Layout"}
			\caption{16 Bit Register Layout} 
			\label{fig:nBitRegister-16-Bit-Layout} 
		\end{figure}
	
		The accumulator register had layout performed next. The circuit is illustrated in Figure \ref{fig:nBitRegister-32-Bit-Layout}. 
	
		\begin{figure}[H] 
			\centering 
			\includegraphics[width=\textwidth,height=\dimexpr\textheight-4\baselineskip-\abovecaptionskip-\belowcaptionskip\relax,keepaspectratio]{"Pictures/nBitRegister 32-Bit Layout"}
			\caption{Accumulator Layout} 
			\label{fig:nBitRegister-32-Bit-Layout} 
		\end{figure}
	
		A multiplexer was needed to swtich between test input and user input, so it was laid out after the rest of the components. The result can be seen in Figure \ref{fig:nBitMux-2to1-Layout}.

		\begin{figure}[H] 
			\centering 
			\includegraphics[width=\textwidth,height=\dimexpr\textheight-4\baselineskip-\abovecaptionskip-\belowcaptionskip\relax,keepaspectratio]{"Pictures/nBitMux_2to1 Layout"}
			\caption{32 Bit Mux 2to1 Layout} 
			\label{fig:nBitMux-2to1-Layout} 
		\end{figure}
	
		
	\subsection{MAC with BIST Layout}
	
		It was found that the circuit could not be constructed structurally with the provided tools. Instead, the circuit was laid out in one block. In order to make the layout possible, many settings were adjusted. The circuit was first auto-instantiated(Figure \ref{fig:Pre-Layout}). 
	
		\begin{figure}[H] 
			\centering 
			\includegraphics[width=\textwidth,height=\dimexpr\textheight-4\baselineskip-\abovecaptionskip-\belowcaptionskip\relax,keepaspectratio]{"Pictures/Pre-Layout"}
			\caption{Pre-Layout} 
			\label{fig:Pre-Layout} 
		\end{figure}
	
		The instantiation could be best described as spaghetti. To organize this pasta, floor-planning was performed. To ensure enough area for routing wires, the area was defined to be 2 when performing the floor plan. Next, standard cells were placed. The cells were initially placed with "random+improve" and optimize. A second cell placement was performed with "initial+improve" and optimize. The second cell placement greatly clean up the circuit as seen in Figure \ref{fig:Std-Cells}.
	
		\begin{figure}[H] 
			\centering 
			\includegraphics[width=\textwidth,height=\dimexpr\textheight-4\baselineskip-\abovecaptionskip-\belowcaptionskip\relax,keepaspectratio]{"Pictures/Std Cells"}
			\caption{Standard Cells Placement} 
			\label{fig:Std-Cells} 
		\end{figure}
	
		After cells placement, ports were placed as close as possible to their sources. Power routing was performed next and the result was recorded in Figure \ref{fig:Power-Route}.
		
		\begin{figure}[H] 
			\centering 
			\includegraphics[width=\textwidth,height=\dimexpr\textheight-4\baselineskip-\abovecaptionskip-\belowcaptionskip\relax,keepaspectratio]{"Pictures/Power Route"}
			\caption{Power Route} 
			\label{fig:Power-Route} 
		\end{figure}
	
		Once power routing was finished, signal routing was performed. Auto-route settings were adjusted to not route with poly silicon as this tends to cause stray gates. Additionally, the following settings were applied to aid in auto routing:
		\begin{itemize}
			\item Varying levels of routing completion time
			\item Slight preference for jogs over via to fill the area.
			\item Rip
			\item Under rip options: 
			\subitem Rips Most Aggressive
			\subitem Automatic Rip Passes
			\subitem Reroute
			\item Under Advanced:
			\subitem Allow all directions for stubs
			\subitem Via Options $>$ Use via generator
		\end{itemize}
	
		Many attempts to route were performed. The working formula consisted of 1 pass of routing with the number of routes turned to the max, and then a second pass consisted of the routes turned to a minimum and a preference for vias instead of jogs. The resulting layout can be seen in Figure \ref{fig:Full-Layout}.
		
		//TODO add pic of failed attempt
		
		\begin{figure}[H] 
			\centering 
			\includegraphics[width=\textwidth,height=\dimexpr\textheight-4\baselineskip-\abovecaptionskip-\belowcaptionskip\relax,keepaspectratio]{"Pictures/Full Layout"}
			\caption{Full Layout} 
			\label{fig:Full-Layout} 
		\end{figure}
	
		A close up view of the layout can be seen in Figure \ref{fig:Full-Layout-Close-Up-View}.
	
		\begin{figure}[H] 
			\centering 
			\includegraphics[width=\textwidth,height=\dimexpr\textheight-4\baselineskip-\abovecaptionskip-\belowcaptionskip\relax,keepaspectratio]{"Pictures/Full Layout Close Up View"}
			\caption{Full Layout Close Up View} 
			\label{fig:Full-Layout-Close-Up-View} 
		\end{figure}
	
		To confirm that routing matched the schematic, an Layout Versus Schematic (LVS) test was performed. The passing test can be seen in Figure \ref{fig:LVS-Pass}. The full report can be seen in Listing \ref{lst:lvs}.
		
		\begin{figure}[H] 
			\centering 
			\includegraphics[width=\textwidth,height=\dimexpr\textheight-4\baselineskip-\abovecaptionskip-\belowcaptionskip\relax,keepaspectratio]{"Pictures/LVS Pass"}
			\caption{Layout Versus Schematics Results} 
			\label{fig:LVS-Pass} 
		\end{figure}
	
	\subsection{Power}
		Power was measured with an Eldo simulation based on the layout. To perform the simulation, a SPICE file was created which can be seen in Listing \ref{lst:power-test-spice}. To measure static power, the average power was measured while the circuit was not active but powered. The static power was measured to be 1.87nW and was recorded in Table \ref{tab:power}. 
		
		Dynamic power was measured by recording the maximum power measured while the circuit was changing as many transistors as possible. To activate as many transistor as possible, multiple, random inputs were supplied to the circuit for a long period of time (2000us). The maximum power was found to be 16.03mW, which was recorded in Table \ref{tab:power}.
		
		
		
		\begin{table}[H]
			\centering
			\caption{Simulated Power for MAC}
			\label{tab:power}
			\begin{tabular}{|cc|}
				\hline
				\textbf{}        & \textbf{Measured Power (W)} \\
				\hline
				\textbf{Static}  & 1.8787E-06           \\
				\textbf{Dynamic} & 1.6029E-02           \\ 
				\hline                     
			\end{tabular}
		\end{table}
	
		It is clear that for this circuit, dynamic power far exceeds the static power. This shows that arithmetic operations draw a lot of power. This is mostly due to their high activity factor and their high speed requirements.  

			

\section{Conclusion}




\section{Appendix}

	\subsection{VHDL}
	
		\lstinputlisting[caption={MAC tb VHDL }\label{lst:MAC-tb-vhd}]{"SourceCode/MAC_tb.vhd"}
		
		\lstinputlisting[caption={ProjectWrapper tb VHDL }\label{lst:ProjectWrapper-tb-vhd}]{"SourceCode/ProjectWrapper_tb.vhd"}
		
		\lstinputlisting[caption={FullAdder VHDL }\label{lst:FullAdder-vhd}]{"SourceCode/FullAdder.vhd"}
		
		\lstinputlisting[caption={FA 1bit VHDL }\label{lst:FA-1bit-vhd}]{"SourceCode/FA_1bit.vhd"}
		
		\lstinputlisting[caption={AND2 VHDL }\label{lst:AND2-vhd}]{"SourceCode/AND2.vhd"}
		
		\lstinputlisting[caption={nBitRegister VHDL }\label{lst:nBitRegister-vhd}]{"SourceCode/nBitRegister.vhd"}
		
		\lstinputlisting[caption={Shifter VHDL }\label{lst:Shifter-vhd}]{"SourceCode/Shifter.vhd"}
		
		\lstinputlisting[caption={TestController VHDL }\label{lst:TestController-vhd}]{"SourceCode/TestController.vhd"}
		
		\lstinputlisting[caption={Subtractor VHDL }\label{lst:Subtractor-vhd}]{"SourceCode/Subtractor.vhd"}
		
		\lstinputlisting[caption={Controller VHDL }\label{lst:Controller-vhd}]{"SourceCode/Controller.vhd"}
		
		\lstinputlisting[caption={ProjectWrapper VHDL }\label{lst:ProjectWrapper-vhd}]{"SourceCode/ProjectWrapper.vhd"}
		
		\lstinputlisting[caption={Counter VHDL }\label{lst:Counter-vhd}]{"SourceCode/Counter.vhd"}
		
		\lstinputlisting[caption={BIST tb VHDL }\label{lst:BIST-tb-vhd}]{"SourceCode/BIST_tb.vhd"}
		
		\lstinputlisting[caption={Multiplier VHDL }\label{lst:Multiplier-vhd}]{"SourceCode/Multiplier.vhd"}
		
		\lstinputlisting[caption={LFSR 32 4 VHDL }\label{lst:LFSR-32-4-vhd}]{"SourceCode/LFSR_32_4.vhd"}
		
		\lstinputlisting[caption={LFSR 8 4 VHDL }\label{lst:LFSR-8-4-vhd}]{"SourceCode/LFSR_8_4.vhd"}
		
		\lstinputlisting[caption={MAC VHDL }\label{lst:MAC-vhd}]{"SourceCode/MAC.vhd"}
		
		\lstinputlisting[caption={MISR 32 4 VHDL }\label{lst:MISR-32-4-vhd}]{"SourceCode/MISR_32_4.vhd"}
		
		\lstinputlisting[caption={nBitRegister tb VHDL }\label{lst:nBitRegister-tb-vhd}]{"SourceCode/nBitRegister_tb.vhd"}
		
		\lstinputlisting[caption={nBitAdder VHDL }\label{lst:nBitAdder-vhd}]{"SourceCode/nBitAdder.vhd"}
		
		\lstinputlisting[caption={nBitRegister 16 VHDL }\label{lst:nBitRegister-16-vhd}]{"SourceCode/nBitRegister_16.vhd"}
		
		\lstinputlisting[caption={MISR 8 4 VHDL }\label{lst:MISR-8-4-vhd}]{"SourceCode/MISR_8_4.vhd"}
		
		\lstinputlisting[caption={nBitMux 2to1 VHDL }\label{lst:nBitMux-2to1-vhd}]{"SourceCode/nBitMux_2to1.vhd"}
		
		\lstinputlisting[caption={ANDADD VHDL }\label{lst:ANDADD-vhd}]{"SourceCode/ANDADD.vhd"}
		
		\lstinputlisting[caption={Ripple Carry FA VHDL }\label{lst:Ripple-Carry-FA-vhd}]{"SourceCode/Ripple_Carry_FA.vhd"}
		
		\lstinputlisting[caption={nBitRegister 32 VHDL }\label{lst:nBitRegister-32-vhd}]{"SourceCode/nBitRegister_32.vhd"}
		
		\lstinputlisting[caption={ALU Wrapper VHDL }\label{lst:ALU-Wrapper-vhd}]{"SourceCode/ALU_Wrapper.vhd"}
		
		\lstinputlisting[caption={nBitAdderSubtractor 16Bit VHDL }\label{lst:nBitAdderSubtractor-16Bit-vhd}]{"SourceCode/nBitAdderSubtractor_16Bit.vhd"}
		
		\lstinputlisting[caption={ALU TESTBENCH VHDL }\label{lst:ALU-TESTBENCH-vhd}]{"SourceCode/ALU_TESTBENCH.vhd"}
		
		\lstinputlisting[caption={Logic Unit VHDL }\label{lst:Logic-Unit-vhd}]{"SourceCode/Logic_Unit.vhd"}
		

	\subsection{Leonardo Scripts}
		
		\lstinputlisting[caption={Multiplier Spectrum Script }\label{lst:Multiplier-script}]{"LeonardoScripts/Multiplier.script"}
		
		\lstinputlisting[caption={LFSR 32 4 Spectrum Script }\label{lst:LFSR-32-4-script}]{"LeonardoScripts/LFSR_32_4.script"}
		
		\lstinputlisting[caption={MISR 32 4 Spectrum Script }\label{lst:MISR-32-4-script}]{"LeonardoScripts/MISR_32_4.script"}
		
		\lstinputlisting[caption={ProjectWrapper Spectrum Script }\label{lst:ProjectWrapper-script}]{"LeonardoScripts/ProjectWrapper.script"}
		
		\lstinputlisting[caption={nBitAdder Spectrum Script }\label{lst:nBitAdder-script}]{"LeonardoScripts/nBitAdder.script"}
		
		\lstinputlisting[caption={nBitAdderSubtractor Spectrum Script }\label{lst:nBitAdderSubtractor-script}]{"LeonardoScripts/nBitAdderSubtractor.script"}
		
		\lstinputlisting[caption={MAC BIST Spectrum Script }\label{lst:MAC-BIST-script}]{"LeonardoScripts/MAC-BIST.script"}
		
		\lstinputlisting[caption={nBitRegister 16 Spectrum Script }\label{lst:nBitRegister-16-script}]{"LeonardoScripts/nBitRegister_16.script"}
		
		\lstinputlisting[caption={MAC Spectrum Script }\label{lst:MAC-script}]{"LeonardoScripts/MAC.script"}
		
		\lstinputlisting[caption={nBitMux 2to1 Spectrum Script }\label{lst:nBitMux-2to1-script}]{"LeonardoScripts/nBitMux_2to1.script"}
		
		\lstinputlisting[caption={nBitRegister 32 Spectrum Script }\label{lst:nBitRegister-32-script}]{"LeonardoScripts/nBitRegister_32.script"}
		
		\lstinputlisting[caption={MAC (copy) Spectrum Script }\label{lst:MAC-(copy)-script}]{"LeonardoScripts/MAC (copy).script"}
		
		
		
	\subsection{Layout Versus Schematic Results}
	
		\lstinputlisting[caption={MAC with BIST LVS Results}\label{lst:lvs}]{"LVS/lvs.report"}
		
	\subsection{SPICE}
	
		\lstinputlisting[caption={layout test SPICE }\label{lst:layout-test-spice}]{"Spice/layout_test.cir"}
		
		\lstinputlisting[caption={power test SPICE }\label{lst:power-test-spice}]{"Spice/power_test.cir"}
		
		

		
\section{References}

	Key, Brandon A. \textit{CMPE-630 Digital IC Design Laboratory Exercise 5 Full-Custom Layout Techniques}. Full-Custom Layout Techniques.
	
	Key, Brandon A. \textit{CMPE-630 Digital IC Design Laboratory Exercise 7 Autolayout Design Techniques (HDL-Layout)}. Autolayout Design Techniques (HDL-Layout).
	
	Key, Brandon A. \textit{CMPE 260 Laboratory Exercise 3 Arithmetic Logic Unit}. CMPE 260 Laboratory Exercise 3 Arithmetic Logic Unit.
	
	Key, Brandon A. \textit{CMPE 260 Laboratory Exercise 5 Binary Multiplier with Built in Self Test}. CMPE 260 Laboratory Exercise 5 Binary Multiplier with Built in Self Test.
	
	//TODO Added Chris's DSD II lab

\end{document}
